\section{结论}

可以看到,理论分析给出的结果为,只有摆长震荡频率$\Omega$使得$\frac{\Omega_{0}}{\Omega}$在$k/2=1/2,1,3/2,\cdots$附近时,参数共振才可能发生,并且发生参数共振的参数空间,受到$\beta$的影响。

模拟计算对参数空间$\{(\omega,\beta)|\omega,\beta\in\mathbb{R}\}$中$\omega$和$\beta$的不同组合进行了模拟,计算比较了不同参数组合下的 PPVL 摆的振幅最大值。可以看到,只有当$\omega=\frac{\Omega_{0}}{\Omega}=\frac{1}{2}$,即摆长变化的频率为单摆固有频率的2倍时,参数共振的效应是显著的,与理论计算符合。

理论计算中引入了稳定性理论,不加证明地使用了弗洛凯定理(Floquet theory)。更加细致的分析可以查阅文后列举的参考文献。模拟计算中未涉及到摆长振幅参数$\epsilon$对参数共振效应的影响,但可以预测,$\omega=\frac{\Omega_{0}}{\Omega}$对参数共振的影响是决定性的,$\epsilon$的大小只会影响参数共振达到幅值的时间。