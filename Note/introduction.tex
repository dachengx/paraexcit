\section{引言}

具有周期性变化摆长的单摆振动问题(Oscillations of a pendulum with periodically variable length (PPVL))问题,是分析力学中的一个经典问题\cite{belyakov_nonlinear_2009}。考虑一维情形,此时系统的拉格朗日函数为:

\begin{align}
    \mathcal{L} &= \frac{1}{2}m\dot{l}^{2}+\frac{1}{2}ml^{2}\dot{\theta}^{2}-mgl(1-\cos\theta)
\end{align}

考虑到系统在此时受到阻尼力形为:

\begin{align}
    f_{damp} &= -\gamma l\dot{\theta}
\end{align}

则运动方程为:

\begin{align}
    \ddot{\theta}+\frac{2}{l}\dot{\theta}\dot{l}+\frac{\gamma}{m}\dot{\theta}+\frac{g}{l}\sin\theta &= 0 \label{move} \\
\end{align}

此处$\dot{}$指对时间$t$求导。

假设单摆长度周期性变化:

\begin{align}
    l &= l_{0} +a\varphi(\Omega t) \label{length}
\end{align}

设$\tau=\Omega t,\epsilon=\frac{a}{l_{0}},\Omega=\sqrt{\frac{g}{l_{0}}},\omega=\frac{\Omega_{0}}{\Omega},\beta=\frac{\gamma}{m\Omega_{0}}$,代入\eqref{length},方程\eqref{move}变为:

\begin{align}
    \ddot{\theta}+\left(\frac{2\epsilon\dot{\varphi}(\tau)}{1+\epsilon\varphi(\tau)}+\beta\omega\right)\dot{\theta}+\frac{\omega^{2}\sin\theta}{1+\epsilon\varphi(\tau)} &= 0
\end{align}

此后$\dot{}$指对$\tau$求导。

PPVL问题中三个关键可变动参数为:激励幅度(excitation amplitude)$\epsilon$,阻尼系数(damping coefficient)$\beta$,频率比$\omega$。

为了表示方便,做变量代换:

\begin{align}
    \theta &= \frac{q}{1+\epsilon\varphi(\tau)}
\end{align}

得到:

\begin{align}
    \ddot{q}+\beta\omega\dot{q}-\frac{\epsilon(\ddot{\varphi}(\tau)+\beta\omega\dot{\varphi}(\tau))}{1+\epsilon\varphi(\tau)}q+\omega^{2}\sin\left(\frac{q}{1+\epsilon\varphi(\tau)}\right) &= 0 \label{set}
\end{align}