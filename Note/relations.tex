\section{解析分析}

\subsection{弗洛凯定理}

可见方程\eqref{set}的平凡解为$q=0$,$q$的稳定性与$\theta$的稳定性相同。根据李雅普诺夫稳定性定理(\href{https://en.wikipedia.org/wiki/Lyapunov_stability}{Lyapunov's theorem on stability})\cite{belyakov_nonlinear_2009},\eqref{set}的稳定性由如下方程决定:

\begin{align}
    \ddot{q}+\beta\omega\dot{q}+\frac{\omega-\epsilon(\ddot{\varphi}(\tau)+\beta\omega\dot{\varphi}(\tau))}{1+\epsilon\varphi(\tau)}q &= 0 \label{stable}
\end{align}

考虑到$\beta$和$\epsilon$均为小量,将\eqref{stable}围绕$\epsilon=0$进行 Taylor 展开,只保留一阶小量:

\begin{align}
    \ddot{q}+\beta\omega\dot{q}+\left[\omega^{2}-\epsilon\left(\ddot{\varphi}(\tau)+\omega^{2}\varphi(\tau)\right)\right]q &=0 \label{hill}
\end{align}

\eqref{hill}为包含阻尼,且周期函数为$-\ddot{\varphi}(\tau)-\omega^{2}\varphi(\tau)$的希尔方程(\href{https://en.wikipedia.org/wiki/Hill_differential_equation}{Hill differential equation})。

下面讨论\label{hill}出现不稳定解(即参数共振解)的条件。

有满足如下常微分方程的线性周期系统:

\begin{align}
    \dot{\mathbf{x}} &= \mathbf{G}(t)\mathbf{x},\,\mathbf{x}\in\mathbb{R}^{m} \label{sys}
\end{align}

其中$\mathbf{G}(t)$是一个大小为$m\times m$的实矩阵,具有周期性:

\begin{align}
    \mathbf{G}(t) &= \mathbf{G}(t+T)
\end{align}

同时若存在两矩阵$\mathbf{X}(t)$和$\mathbf{Y}(t)$满足:

\begin{align}
    \dot{\mathbf{X}} &= \mathbf{G}(t)\mathbf{X},\,\mathbf{X}(0)=\mathbf{I} \label{x} \\ 
    \dot{\mathbf{Y}} &= -\mathbf{G}^{T}(t)\mathbf{Y},\,\mathbf{Y}(0)=\mathbf{I} \label{y} \\
    \mathbf{I} &= \mathbf{Y}^{T}(t)\mathbf{X}(t) \label{xy}
\end{align}

设$\rho$为 Floquet matrix $\mathbf{F}=\mathbf{X}(T)$的特征值,称为 Floquet multiplier。则有如下定理:

\begin{theorem}[\href{https://en.wikipedia.org/wiki/Floquet_theory}{弗洛凯定理(Floquet theory)} \cite{seyranian_multiparameter_nodate}]
\label{Floquet}
\,\newline
Linear periodic system \eqref{sys} is stable (all the solution $\mathbf{x}$ are bounded as $t\rightarrow\infty$) if and only if $|\rho|\leq1$ for all the multipliers of the Floquet matrix, and the multipliers with the unit absolute value $|\rho|=1$ are simple or semi-simple. 
\end{theorem}

其中semi-simple 指特征值的几何重数(Geometric multiplicity)等于代数重数(Algebraic multiplicity)。

\subsection{自然共振频率}

考察方程,对应的$\mathbf{x}$和$\mathbf{G}(t)$为:

\begin{align}
    \mathbf{x} &= \binom{q}{\dot{q}}, \,
    \mathbf{G}(t) = 
    \begin{pmatrix}
        0 & 1 \\ 
        -\omega^{2}+\epsilon\left(\ddot{\varphi}(\tau)+\omega^{2}\varphi(\tau)\right) & \beta\omega
    \end{pmatrix}
\end{align}

考虑到$\beta$和$\epsilon$均较小,首先做近似$\beta\approx0,\epsilon\approx0$,得到:

\begin{align}
    \mathbf{G}(t) = 
    \begin{pmatrix}
        0 & 1 \\ 
        -\omega^{2}+\epsilon\left(\ddot{\varphi}(\tau)+\omega^{2}\varphi(\tau)\right) & \beta\omega
    \end{pmatrix}\approx
    \begin{pmatrix}
        0 & 1 \\ 
        -\omega^{2} & 0
    \end{pmatrix}
\end{align}

那么,我们找到满足\eqref{x},\eqref{y},\eqref{xy}的$\mathbf{X}(t)$和$\mathbf{Y}(t)$为:

\begin{align}
    \mathbf{X}(t) &= 
    \begin{pmatrix}
        \cos\omega t & \frac{\sin\omega t}{\omega} \\ 
        -\omega\sin\omega t & \cos\omega t
    \end{pmatrix} \\
    \mathbf{Y}(t) &= 
    \begin{pmatrix}
        \cos\omega t & \omega\sin\omega t \\ 
        -\frac{\sin\omega t}{\omega} & \cos\omega t
    \end{pmatrix} \\
\end{align}

则 Floquet matrix 为:
\begin{align}
    \mathbf{F}_{0} &= \mathbf{X}(2\pi) =
    \begin{pmatrix}
        \cos2\pi\omega & \frac{\sin2\pi\omega}{\omega} \\ 
        -\omega\sin2\pi\omega & \cos2\pi\omega
    \end{pmatrix}
\end{align}

求$\mathbf{F}_{0}$的本征值:

\begin{align}
    \rho^{a,b} &= \cos2\pi\omega\pm i\sin2\pi\omega
\end{align}

当对于任意$k=1,2,\cdots$,$\omega\neq\frac{k}{2}$,则两个乘子$\rho^{a},\rho^{a}$为共轭的 simple 特征值,位分别位于复平面的单位圆上。对于$\omega$的取值:

\begin{align}
    \omega_{0} &= \frac{k}{2},\,k=1,2,\cdot \label{resonance}
\end{align}

两个乘子为:

\begin{align}
    \rho^{a} &= \rho^{b} = (-1)^{k}
\end{align}

对应的 Floquet matrix 为$\mathbf{F}_{0}=(-1)^{k}\mathbf{I}$,两个乘子$\rho^{a},\rho^{a}$为 semi-simple 。\eqref{resonance}称为自然频率的共振值(resonance (critical) values of the natural frequency)。

上述结果是近似$\beta\approx0,\epsilon\approx0$情况下得到的,当考虑到$\beta$和$\epsilon$的非0值,对Floquet matrix $\mathbf{F}_{0}$做更加细致的分析得到\cite{seyranian_multiparameter_nodate},只有在$omega_{0}=\frac{k}{2},\,k=1,2,\cdot$附近,即自然频率的共振值附近时,Floquet matrix 的特征值的模长$|\rho|$才可能是不小于0,即满足\eqref{Floquet}中的不稳定条件,系统才可能发生参数共振。

\subsection{自然共振频率的2倍频关系}

基于上述过程可以分析得到\cite{seyranian_multiparameter_nodate},在$\epsilon,\beta,\omega$的三维参数空间中,PPVL系统的稳定性由一个半圆椎描述:

\begin{align}
    \left(\frac{\beta}{2}\right)^{2} + \left(\frac{2\omega}{k}-1\right)^{2} &< \left(a_{k}^{2}+b_{k}^{2}\right)(\frac{3\epsilon}{4})^{2} \label{cone}
\end{align}

其中$a_{k}$和$b_{k}$为周期函数$\varphi(\tau)$的 Fourier 分解系数:

\begin{align}
    a_{k} &= \frac{1}{\pi}\int_{0}^{2\pi}\varphi(\tau)\cos k\pi\mathrm{d}\tau,\,b_{k} = \frac{1}{\pi}\int_{0}^{2\pi}\varphi(\tau)\sin k\pi\mathrm{d}\tau
\end{align}

对于$\varphi(\tau)=\cos(\tau),k=1$,\eqref{cone}变为:

\begin{align}
    \left(\frac{\beta}{2}\right)^{2} + \left(\frac{2\omega}{k}-1\right)^{2} &< (\frac{3\epsilon}{4})^{2}
\end{align}

对于其他$k\neq1$,因为$a_{k}=b_{k}=0$,\eqref{cone}退化成一个点($\beta=0,\omega=\frac{k}{2}$)。在实际运动中,由于\eqref{cone}的得出经过了几步近似,对于$k\neq1$的频率,仍然可以观察到小幅度的震荡,但远比$k=1$时的共振幅度小。